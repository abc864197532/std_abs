% Source: waynedisonitau123 & 8BQube

\subsubsection{Kirchhoff's Theorem}
Denote $L$ be a $n \times n$ matrix as the Laplacian matrix of graph $G$, where $L_{ii} = d(i)$, $L_{ij} = -c$ where $c$ is the number of edge $(i, j)$ in $G$.
\begin{itemize}
    \item The number of undirected spanning in $G$ is $\lvert \det(\tilde{L}_{11}) \rvert$.
    \item The number of directed spanning tree rooted at $r$ in $G$ is $\lvert \det(\tilde{L}_{rr}) \rvert$.
\end{itemize}

\subsubsection{Tutte's Matrix}
Let $D$ be a $n \times n$ matrix, where $d_{ij} = x_{ij}$ ($x_{ij}$ is chosen uniformly at random) if $i < j$ and $(i, j) \in E$, otherwise $d_{ij} = -d_{ji}$. $\frac{rank(D)}{2}$ is the maximum matching on $G$.

\subsubsection{Cayley's Formula}
\begin{itemize}
  \item Given a degree sequence $d_1, d_2, \ldots, d_n$ for each \textit{labeled} vertices, there are $$\frac{(n - 2)!}{(d_1 - 1)!(d_2 - 1)!\cdots(d_n - 1)!}$$ spanning trees.
  \item Let $T_{n, k}$ be the number of \textit{labeled} forests on $n$ vertices with $k$ components, such that vertex $1, 2, \ldots, k$ belong to different components. Then $T_{n, k} = kn^{n - k - 1}$.
\end{itemize}

\subsubsection{Erdős–Gallai Theorem}
A sequence of non-negative integers $d_1 \geq d_2 \geq \ldots \geq d_n$ can be represented as the degree sequence of a finite simple graph on $n$ vertices if and only if $d_1 + d_2 + \ldots + d_n$ is even and
$$ \sum_{i = 1}^{k}d_i \leq k(k - 1) + \sum_{i = k + 1}^{n}\min(d_i, k) $$
holds for all $1 \leq k \leq n$.

\subsubsection{Burnside's Lemma}
Let $X$ be a set and $G$ be a group that acts on $X$.
For $g \in G$, denote by $X^g$ the elements fixed by $g$:
\[
X^g = \{ x \in X \mid gx \in X \}
\]
Then
\[
|X/G| = \frac{1}{|G|} \sum_{g \in G} |X^g|.
\]

\subsubsection{Gale–Ryser theorem}
A pair of sequences of nonnegative integers $a_1\ge\cdots\ge a_n$ and $b_1,\ldots,b_n$ is bigraphic if and only if $\displaystyle\sum_{i=1}^n a_i=\displaystyle\sum_{i=1}^n b_i$ and $\displaystyle\sum_{i=1}^k a_i\le \displaystyle\sum_{i=1}^n\min(b_i,k)$ holds for every $1\le k\le n$.

\subsubsection{Fulkerson–Chen–Anstee theorem}
A sequence $(a_1,b_1),\ldots,(a_n,b_n)$ of nonnegative integer pairs with $a_1\ge\cdots\ge a_n$ is digraphic if and only if $\displaystyle\sum_{i=1}^n a_i=\displaystyle\sum_{i=1}^n b_i$ and $\displaystyle\sum_{i=1}^k a_i\le \displaystyle\sum_{i=1}^k\min(b_i,k-1)+\displaystyle\sum_{i=k+1}^n\min(b_i,k)$ holds for every $1\le k\le n$.

\subsubsection{Möbius inversion formula}
\begin{itemize}
    \itemsep-0.5em
  \item $f(n)=\sum_{d\mid n}g(d)\Leftrightarrow g(n)=\sum_{d\mid n}\mu(d)f(\frac{n}{d})$
  \item $f(n)=\sum_{n\mid d}g(d)\Leftrightarrow g(n)=\sum_{n\mid d}\mu(\frac{d}{n})f(d)$
\end{itemize}

\subsubsection{Spherical cap}
\begin{itemize}
    \itemsep-0.5em
  \item A portion of a sphere cut off by a plane.
  \item $r$: sphere radius, $a$: radius of the base of the cap, $h$: height of the cap, $\theta$: $\arcsin(a/r)$.
  \item Volume $=\pi h^2(3r-h)/3=\pi h(3a^2+h^2)/6=\pi r^3(2+\cos\theta)(1-\cos\theta)^2/3$.
  \item Area $=2\pi rh=\pi(a^2+h^2)=2\pi r^2(1-\cos\theta)$.
\end{itemize}

\subsubsection{Chinese Remainder Theorem}
\begin{itemize}
    \itemsep-0.5em
  \item $x \equiv a_i \pmod {m_i}$
  \item $M = \prod m_i, M_i = M / m_i$
  \item $t_iM_i \equiv 1 \pmod {m_i}$
  \item $x = \sum a_it_iM_i \pmod M$
\end{itemize}