\documentclass[a4paper,10pt,twocolumn,oneside]{article}
\setlength{\columnsep}{10pt}                                                                    %兩欄模式的間距
\setlength{\columnseprule}{0pt}                                                                %兩欄模式間格線粗細

\usepackage{amsthm}								%定義,例題
\usepackage{amssymb}
%\usepackage[margin=2cm]{geometry}
\usepackage{fontspec}								%設定字體
\usepackage{color}
\usepackage[x11names]{xcolor}
\usepackage{listings}								%顯示code用的
\usepackage[Glenn]{fncychap}						%排版,頁面模板
\usepackage{fancyhdr}								%設定頁首頁尾
\usepackage{graphicx}								%Graphic
\usepackage{enumerate}
\usepackage{titlesec}
\usepackage{amsmath}
% \usepackage[CheckSingle, CJKmath]{xeCJK}
% \usepackage{CJKulem}

%\usepackage[T1]{fontenc}
\usepackage{amsmath, courier, listings, fancyhdr, graphicx}
\topmargin=0pt
\headsep=5pt
\textheight=780pt
\footskip=0pt
\voffset=-40pt
\textwidth=545pt
\marginparsep=0pt
\marginparwidth=0pt
\marginparpush=0pt
\oddsidemargin=0pt
\evensidemargin=0pt
\hoffset=-42pt

\titlespacing\subsection{0pt}{5pt plus 4pt minus 2pt}{0pt plus 2pt minus 2pt}


%\renewcommand\listfigurename{圖目錄}
%\renewcommand\listtablename{表目錄} 

%%%%%%%%%%%%%%%%%%%%%%%%%%%%%

\setmainfont{Consolas}				%主要字型
%\setmonofont{Monaco}				%主要字型
\setmonofont{Consolas}
% \setCJKmainfont{Noto Sans CJK TC}
% \setCJKmainfont{Consolas}			%中文字型
%\setmainfont{sourcecodepro}
\XeTeXlinebreaklocale "zh"						%中文自動換行
\XeTeXlinebreakskip = 0pt plus 1pt				%設定段落之間的距離
\setcounter{secnumdepth}{3}						%目錄顯示第三層

%%%%%%%%%%%%%%%%%%%%%%%%%%%%%
\makeatletter
\lst@CCPutMacro\lst@ProcessOther {"2D}{\lst@ttfamily{-{}}{-{}}}
\@empty\z@\@empty
\makeatother
\lstset{											% Code顯示
language=C++,										% the language of the code
basicstyle=\footnotesize\ttfamily, 						% the size of the fonts that are used for the code
%numbers=left,										% where to put the line-numbers
numberstyle=\footnotesize,						% the size of the fonts that are used for the line-numbers
stepnumber=1,										% the step between two line-numbers. If it's 1, each line  will be numbered
numbersep=5pt,										% how far the line-numbers are from the code
backgroundcolor=\color{white},					% choose the background color. You must add \usepackage{color}
showspaces=false,									% show spaces adding particular underscores
showstringspaces=false,							% underline spaces within strings
showtabs=false,									% show tabs within strings adding particular underscores
frame=false,											% adds a frame around the code
tabsize=2,											% sets default tabsize to 2 spaces
captionpos=b,										% sets the caption-position to bottom
breaklines=true,									% sets automatic line breaking
breakatwhitespace=false,							% sets if automatic breaks should only happen at whitespace
escapeinside={\%*}{*)},							% if you want to add a comment within your code
morekeywords={constexpr},									% if you want to add more keywords to the set
keywordstyle=\bfseries\color{Blue1},
commentstyle=\itshape\color{Red4},
stringstyle=\itshape\color{Green4},
}

%%%%%%%%%%%%%%%%%%%%%%%%%%%%%

\begin{document}
\pagestyle{fancy}
\fancyfoot{}
%\fancyfoot[R]{\includegraphics[width=20pt]{ironwood.jpg}}
\fancyhead[L]{National Taiwan University std\_abs}
\fancyhead[R]{\thepage}
\renewcommand{\headrulewidth}{0.4pt}
\renewcommand{\contentsname}{Contents} 

\scriptsize
\tableofcontents
%%%%%%%%%%%%%%%%%%%%%%%%%%%%%

%\newpage

\section{Basic}
\subsection{Shell Script}
\lstinputlisting{Basic/Script.sh}
\subsection{Default Code}
\lstinputlisting{Basic/DefaultCode.cpp}
% \subsection{Testing Todo List*}
% \lstinputlisting{Basic/Testing.txt}
\subsection{Increase Stack Size}
\lstinputlisting{Basic/IncreaseStackSize.cpp}
\subsection{Debug Macro}
\lstinputlisting{Basic/DebugMacro.cpp}
% \subsection{Stress Test Shell*}
% \lstinputlisting{Basic/Stress.sh}
\subsection{Pragma / FastIO}
\lstinputlisting{Basic/Fast.cpp}
\subsection{Divide*}
\lstinputlisting{Basic/Div.cpp}

\section{Data Structure}
\subsection{Leftist Tree}
\lstinputlisting{DataStructure/LeftistTree.cpp}
\subsection{Splay Tree}
\lstinputlisting{DataStructure/Splay.cpp}
\subsection{Link Cut Tree}
\lstinputlisting{DataStructure/LinkCutTree.cpp}
\subsection{Treap}
\lstinputlisting{DataStructure/Treap.cpp}
% \subsection{Persistent Segment Tree*}
% \lstinputlisting{DataStructure/PersistentSeg.cpp}
\subsection{2D Segment Tree*}
\lstinputlisting{DataStructure/2DSeg.cpp} % Stress Test with Brute Force
% \subsection{Zkw*}
% \lstinputlisting{DataStructure/Zkw.cpp}
% \subsection{Chtholly Tree*}
% \lstinputlisting{DataStructure/ChthollyTree.cpp}
\subsection{Range Set*}
\lstinputlisting{DataStructure/RangeSet.cpp} % CF 981G, ABC 255H
\subsection{vEB Tree*}
\lstinputlisting{DataStructure/vEBTree.cpp} % Library Checker "Predecessor Problem"
% \subsection{Incremental Min Sum*}
% \lstinputlisting{DataStructure/IncrementalMinSum.cpp}

\section{Flow / Matching}
\subsection{Dinic}
\lstinputlisting{FlowMatching/Dinic.cpp}
\subsection{Min Cost Max Flow}
\lstinputlisting{FlowMatching/MCMF.cpp}
\subsection{Kuhn Munkres}
\lstinputlisting{FlowMatching/KM.cpp} % TIOJ 2197
\subsection{SW Min Cut}
\lstinputlisting{FlowMatching/SW.cpp} % NTUJ 1878
\subsection{Gomory Hu Tree}
\lstinputlisting{FlowMatching/GomoryHu.cpp} % CF 343E
\subsection{Blossom}
\lstinputlisting{FlowMatching/GeneralGraphMatching.cpp} % Library Checker "Matching on General Graph"
\subsection{Weighted Blossom}
\lstinputlisting{FlowMatching/WeightGeneralGraphMatching.cpp} % Library Checker "General Weighted Matching"
\subsection{Flow Model}
% \normalsize
\begin{itemize}
    \itemsep-0.3em
    \item Maximum/Minimum flow with lower bound / Circulation problem
    \vspace{-1em}
    \begin{enumerate}
        \itemsep-0.3em
        \item Construct super source $S$ and sink $T$.
        \item For each edge $(x, y, l, u)$, connect $x \rightarrow y$ with capacity $u - l$.
        \item For each vertex $v$, denote by $in(v)$ the difference between the sum of incoming lower bounds and the sum of outgoing lower bounds.
        \item If $in(v) > 0$, connect $S \rightarrow v$ with capacity $in(v)$, otherwise, connect $v \rightarrow T$ with capacity $-in(v)$.
        \begin{itemize}
            \itemsep-0.2em
            \item To maximize, connect $t \rightarrow s$ with capacity $\infty$ (skip this in circulation problem), and let $f$ be the maximum flow from $S$ to $T$. If $f \neq \sum_{v \in V, in(v) > 0}{in(v)}$, there's no solution. Otherwise, the maximum flow from $s$ to $t$ is the answer.
            \item To minimize, let $f$ be the maximum flow from $S$ to $T$. Connect $t \rightarrow s$ with capacity $\infty$ and let the flow from $S$ to $T$ be $f^\prime$. If $f + f^\prime \neq \sum_{v \in V, in(v) > 0}{in(v)}$, there's no solution. Otherwise, $f^\prime$ is the answer.
        \end{itemize}
        \item The solution of each edge $e$ is $l_e + f_e$, where $f_e$ corresponds to the flow of edge $e$ on the graph.
    \end{enumerate}
    \item Construct minimum vertex cover from maximum matching $M$ on bipartite graph $(X, Y)$
    \vspace{-1em}
    \begin{enumerate}
        \itemsep-0.3em
        \item Redirect every edge: $y \rightarrow x$ if $(x, y) \in M$, $x \rightarrow y$ otherwise.
        \item DFS from unmatched vertices in $X$.
        \item $x \in X$ is chosen iff $x$ is unvisited.
        \item $y \in Y$ is chosen iff $y$ is visited.
    \end{enumerate}
    \item Minimum cost cyclic flow
    \vspace{-0.5em}
    \begin{enumerate}
        \itemsep-0.3em
        \item Consruct super source $S$ and sink $T$
        \item For each edge $(x, y, c)$, connect $x \rightarrow y$ with $(cost, cap) = (c, 1)$ if $c > 0$, otherwise connect $y \rightarrow x$ with $(cost, cap) = (-c, 1)$
        \item For each edge with $c < 0$, sum these cost as $K$, then increase $d(y)$ by 1, decrease $d(x)$ by 1
        \item For each vertex $v$ with $d(v) > 0$, connect $S \rightarrow v$ with $(cost, cap) = (0, d(v))$
        \item For each vertex $v$ with $d(v) < 0$, connect $v \rightarrow T$ with $(cost, cap) = (0, -d(v))$
        \item Flow from $S$ to $T$, the answer is the cost of the flow $C + K$
    \end{enumerate}
    \item Maximum density induced subgraph
    \vspace{-1em}
    \begin{enumerate}
        \itemsep-0.3em
        \item Binary search on answer, suppose we're checking answer $T$
        \item Construct a max flow model, let $K$ be the sum of all weights
        \item Connect source $s \rightarrow v$, $v \in G$ with capacity $K$
        \item For each edge $(u, v, w)$ in $G$, connect $u \rightarrow v$ and $v \rightarrow u$ with capacity $w$
        \item For $v \in G$, connect it with sink $v \rightarrow t$ with capacity $K + 2T - (\sum_{e \in E(v)}{w(e)}) - 2w(v)$
        \item $T$ is a valid answer if the maximum flow $f < K \lvert V \rvert$
    \end{enumerate}
    \item Minimum weight edge cover
    \vspace{-1em}
    \begin{enumerate}
        \itemsep-0.3em
      \item For each $v \in V$ create a copy $v^\prime$, and connect $u^\prime \to v^\prime$ with weight $w(u, v)$.
      \item Connect $v \to v^\prime$ with weight $2\mu(v)$, where $\mu(v)$ is the cost of the cheapest edge incident to $v$.
      \item Find the minimum weight perfect matching on $G^\prime$.
    \end{enumerate}
    \item Project selection problem
    \vspace{-1em}
    \begin{enumerate}
      \itemsep-0.3em
      \item If $p_v > 0$, create edge $(s, v)$ with capacity $p_v$; otherwise, create edge $(v, t)$ with capacity $-p_v$.
      \item Create edge $(u, v)$ with capacity $w$ with $w$ being the cost of choosing $u$ without choosing $v$.
      \item The mincut is equivalent to the maximum profit of a subset of projects.
    \end{enumerate}
    \item 0/1 quadratic programming
    \vspace{-1em}
    \[ \sum_x{c_xx} + \sum_y{c_y\bar{y}} + \sum_{xy}c_{xy}x\bar{y} + \sum_{xyx^\prime y^\prime}c_{xyx^\prime y^\prime}(x\bar{y} + x^\prime\bar{y^\prime}) \]
    can be minimized by the mincut of the following graph:
    \begin{enumerate}
      \itemsep-0.3em
      \item Create edge $(x, t)$ with capacity $c_x$ and create edge $(s, y)$ with capacity $c_y$.
      \item Create edge $(x, y)$ with capacity $c_{xy}$.
      \item Create edge $(x, y)$ and edge $(x^\prime, y^\prime)$ with capacity $c_{xyx^\prime y^\prime}$.
    \end{enumerate}
\end{itemize}

\section{Graph}
\subsection{Binary Lifting}
\lstinputlisting{Graph/BinaryLifting.cpp}
\subsection{Heavy-Light Decomposition}
\lstinputlisting{Graph/HeavyLightDecomp.cpp}
\subsection{Centroid Decomposition}
\lstinputlisting{Graph/CentroidDecomp.cpp}
\subsection{Edge BCC}
\lstinputlisting{Graph/EdgeBCC.cpp} % CF 1000E, Library Checker
\subsection{Vertex BCC / Round Square Tree}
\lstinputlisting{Graph/RoundSquareTree.cpp} % CF 962F, Library Checker
\subsection{SCC / 2SAT}
\lstinputlisting{Graph/2SAT.cpp} % CF 1657F, Library Checker
% \subsection{Negative Cycle*}
% \lstinputlisting{Graph/NegativeCycle.cpp}
\subsection{Virtual Tree}
\lstinputlisting{Graph/VirtualTree.cpp}
\subsection{Directed MST} % Almost Copy from ckiseki
\lstinputlisting{Graph/DirectedMST.cpp} % CF 100307D, Library Checker
\subsection{Dominator Tree}
\lstinputlisting{Graph/DominatorTree.cpp} % Library Checker
\subsection{Vizing}
\lstinputlisting{Graph/Vizing.cpp}

\section{String}
\subsection{Aho–Corasick Automaton}
\lstinputlisting{String/AC.cpp}
\subsection{KMP Algorithm}
\lstinputlisting{String/KMP.cpp}
\subsection{Z Algorithm}
\lstinputlisting{String/Z.cpp} % Library Checker
\subsection{Manacher}
\lstinputlisting{String/Manacher.cpp} % Library Checker
\subsection{Suffix Array}
\lstinputlisting{String/SA.cpp} % Library Checker
\subsection{SAIS}
\lstinputlisting{String/SAIS.cpp} % Library Checker
\subsection{Suffix Automaton}
\lstinputlisting{String/SAM.cpp}
\subsection{Minimum Rotation}
\lstinputlisting{String/MinRotation.cpp} % NTUJ 1751
\subsection{Palindrome Tree}
\lstinputlisting{String/PalindromeTree.cpp}
\subsection{Main Lorentz}
\lstinputlisting{String/MainLorentz.cpp} % CF 1043G, Library Checker

\section{Math}
% \subsection{Fraction*}
% \lstinputlisting{Math/Fraction.cpp}
\subsection{Miller Rabin / Pollard Rho}
\lstinputlisting{Math/MillerRabinPollardRho.cpp} % Library Checker
\subsection{Ext GCD}
\lstinputlisting{Math/Extgcd.cpp}
\subsection{Chinese Remainder Theorem} % Stress Test
\lstinputlisting{Math/CRT.cpp}
\subsection{PiCount}
\lstinputlisting{Math/PiCount.cpp} % Library Checker
\subsection{Linear Function Mod Min} 
\lstinputlisting{Math/LinearFuncModMin.cpp} % Library Checker
\subsection{Determinant*}
\lstinputlisting{Math/Determinant.cpp} % Library Checker
\subsection{Floor Sum}
\lstinputlisting{Math/FloorSum.cpp} % ABC 313G, Library Checker
\subsection{Quadratic Residue}
\lstinputlisting{Math/QuadraticResidue.cpp} % Library Checker
\subsection{Simplex}
\lstinputlisting{Math/Simplex.cpp} % CF 101161J
\subsection{Berlekamp Massey}
\lstinputlisting{Math/BerlekampMassey.cpp} % Library Checker
\subsection{Linear Programming Construction}
% \normalsize
Standard form: maximize $\mathbf{c}^T\mathbf{x}$ subject to $A\mathbf{x} \leq \mathbf{b}$ and $\mathbf{x} \geq 0$. \\
Dual LP: minimize $\mathbf{b}^T\mathbf{y}$ subject to $A^T\mathbf{y} \geq \mathbf{c}$ and $\mathbf{y} \geq 0$. \\
$\bar{\mathbf{x}}$ and $\bar{\mathbf{y}}$ are optimal if and only if for all $i \in [1, n]$, either $\bar{x}_i = 0$ or $\sum_{j=1}^{m}A_{ji}\bar{y}_j = c_i$ holds and for all $i \in [1, m]$ either $\bar{y}_i = 0$ or $\sum_{j=1}^{n}A_{ij}\bar{x}_j = b_j$ holds.

\begin{enumerate}
    \itemsep-0.5em
    \item In case of minimization, let $c^\prime_i = -c_i$
    \item $\sum_{1 \leq i \leq n}{A_{ji}x_i} \geq b_j \rightarrow \sum_{1 \leq i \leq n}{-A_{ji}x_i} \leq -b_j$
    \item $\sum_{1 \leq i \leq n}{A_{ji}x_i} = b_j$ 
        \vspace{-0.5em}
        \begin{itemize}
            \itemsep-0.5em
            \item $\sum_{1 \leq i \leq n}{A_{ji}x_i} \leq b_j$
            \item $\sum_{1 \leq i \leq n}{A_{ji}x_i} \geq b_j$
        \end{itemize}
    \item If $x_i$ has no lower bound, replace $x_i$ with $x_i - x_i^\prime$
\end{enumerate}
\subsection{Euclidean}
$m = \lfloor\frac{an + b}{c}\rfloor$

% $$ \begin{aligned}
%   f(a, b, c, n) &= \sum_{i = 0}^{n}\lfloor\frac{ai + b}{c}\rfloor \\
%   &= \begin{cases} 
%     \lfloor\frac{a}{c}\rfloor \cdot \frac{n(n + 1)}{2} + \lfloor\frac{b}{c}\rfloor \cdot (n + 1) \\ + f(a\text{ mod } c, b\text{ mod } c, c, n), & a \geq c \lor b \geq c \\ 
%     0, & n < 0 \lor a = 0 \\
%     nm - f(c, c - b - 1, a, m - 1), & \text{otherwise} 
%   \end{cases} 
% \end{aligned} $$
$$ \begin{aligned}
  g(a, b, c, n) &= \sum_{i = 0}^{n}i\lfloor\frac{ai + b}{c}\rfloor \\
  &= \begin{cases}
    \lfloor{\frac{a}{c}}\rfloor \cdot \frac{n(n + 1)(2n + 1)}{6} + \lfloor\frac{b}{c}\rfloor \cdot \frac{n(n + 1)}{2} \\ + g(a\text{ mod } c, b\text{ mod } c, c, n), & a \geq c \lor b \geq c \\
    0, & n < 0 \lor a = 0 \\
    \frac{1}{2} \cdot (n(n + 1)m - f(c, c - b - 1, a, m - 1) \\ - h(c, c - b - 1, a, m - 1)), & \text{otherwise}
  \end{cases}
\end{aligned} $$
$$ \begin{aligned}
  h(a, b, c, n) &= \sum_{i = 0}^{n}\lfloor\frac{ai + b}{c}\rfloor^2 \\
  &= \begin{cases}
    \lfloor\frac{a}{c}\rfloor^2 \cdot \frac{n(n + 1)(2n + 1)}{6} + \lfloor\frac{b}{c}\rfloor^2 \cdot (n + 1) \\ + \lfloor\frac{a}{c}\rfloor \cdot \lfloor\frac{b}{c}\rfloor \cdot n(n + 1) \\ + h(a\text{ mod } c, b\text{ mod } c, c, n) \\ + 2\lfloor\frac{a}{c}\rfloor \cdot g(a\text{ mod } c, b\text{ mod } c, c, n) \\ + 2\lfloor\frac{b}{c}\rfloor \cdot f(a\text{ mod } c, b\text{ mod } c, c, n), & a \geq c \lor b \geq c \\
    0, & n < 0 \lor a = 0 \\
    nm(m + 1) - 2g(c, c - b - 1, a, m - 1) \\ - 2f(c, c - b - 1, a, m - 1) - f(a, b, c, n), & \text{otherwise}
  \end{cases}
\end{aligned} $$

\subsection{Theorem}
% Source: waynedisonitau123 & 8BQube

\begin{itemize}
\item Kirchhoff's Theorem

Denote $L$ be a $n \times n$ matrix as the Laplacian matrix of graph $G$, where $L_{ii} = d(i)$, $L_{ij} = -c$ where $c$ is the number of edge $(i, j)$ in $G$.
\begin{itemize}
    \item The number of undirected spanning in $G$ is $\lvert \det(\tilde{L}_{11}) \rvert$.
    \item The number of directed spanning tree rooted at $r$ in $G$ is $\lvert \det(\tilde{L}_{rr}) \rvert$.
\end{itemize}

\item Tutte's Matrix

Let $D$ be a $n \times n$ matrix, where $d_{ij} = x_{ij}$ ($x_{ij}$ is chosen uniformly at random) if $i < j$ and $(i, j) \in E$, otherwise $d_{ij} = -d_{ji}$. $\frac{rank(D)}{2}$ is the maximum matching on $G$.

% \item Cayley's Formula

% \begin{itemize}
%   \item Given a degree sequence $d_1, d_2, \ldots, d_n$ for each \textit{labeled} vertices, there are $$\frac{(n - 2)!}{(d_1 - 1)!(d_2 - 1)!\cdots(d_n - 1)!}$$ spanning trees.
%   \item Let $T_{n, k}$ be the number of \textit{labeled} forests on $n$ vertices with $k$ components, such that vertex $1, 2, \ldots, k$ belong to different components. Then $T_{n, k} = kn^{n - k - 1}$.
% \end{itemize}

\item Erdős–Gallai Theorem

A sequence of non-negative integers $d_1 \geq d_2 \geq \ldots \geq d_n$ can be represented as the degree sequence of a finite simple graph on $n$ vertices if and only if $d_1 + d_2 + \ldots + d_n$ is even and
$$ \sum_{i = 1}^{k}d_i \leq k(k - 1) + \sum_{i = k + 1}^{n}\min(d_i, k) $$
holds for all $1 \leq k \leq n$.

\item Burnside's Lemma

Let $X$ be a set and $G$ be a group that acts on $X$.
For $g \in G$, denote by $X^g$ the elements fixed by $g$:
\[
X^g = \{ x \in X \mid gx \in X \}
\]
Then
\[
|X/G| = \frac{1}{|G|} \sum_{g \in G} |X^g|.
\]

\item Gale–Ryser theorem

A pair of sequences of nonnegative integers $a_1\ge\cdots\ge a_n$ and $b_1,\ldots,b_n$ is bigraphic if and only if $\displaystyle\sum_{i=1}^n a_i=\displaystyle\sum_{i=1}^n b_i$ and $\displaystyle\sum_{i=1}^k a_i\le \displaystyle\sum_{i=1}^n\min(b_i,k)$ holds for every $1\le k\le n$. Sequences $a$ and $b$ called bigraphic if there is a labeled simple bipartite graph such that $a$ and $b$ is the degree sequence of this bipartite graph.

\item Fulkerson–Chen–Anstee theorem

A sequence $(a_1,b_1),\ldots,(a_n,b_n)$ of nonnegative integer pairs with $a_1\ge\cdots\ge a_n$ is digraphic if and only if $\displaystyle\sum_{i=1}^n a_i=\displaystyle\sum_{i=1}^n b_i$ and $\displaystyle\sum_{i=1}^k a_i\le \displaystyle\sum_{i=1}^k\min(b_i,k-1)+\displaystyle\sum_{i=k+1}^n\min(b_i,k)$ holds for every $1\le k\le n$. Sequences $a$ and $b$ called digraphic if there is a labeled simple directed graph such that each vertex $v_i$ has indegree $a_i$ and outdegree $b_i$.

\item Pick's theorem

For simple polygon, when points are all integer, we have $A=\text{\#\{lattice points in the interior\}} + \frac{\text{\#\{lattice points on the boundary\}}}{2} - 1$

% \item Möbius inversion formula

% \begin{itemize}
%     \itemsep-0.5em
%   \item $f(n)=\sum_{d\mid n}g(d)\Leftrightarrow g(n)=\sum_{d\mid n}\mu(d)f(\frac{n}{d})$
%   \item $f(n)=\sum_{n\mid d}g(d)\Leftrightarrow g(n)=\sum_{n\mid d}\mu(\frac{d}{n})f(d)$
% \end{itemize}

\item Spherical cap

\begin{itemize}
    \itemsep-0.5em
  \item A portion of a sphere cut off by a plane.
  \item $r$: sphere radius, $a$: radius of the base of the cap, $h$: height of the cap, $\theta$: $\arcsin(a/r)$.
  \item Volume $=\pi h^2(3r-h)/3=\pi h(3a^2+h^2)/6=\pi r^3(2+\cos\theta)(1-\cos\theta)^2/3$.
  \item Area $=2\pi rh=\pi(a^2+h^2)=2\pi r^2(1-\cos\theta)$.
\end{itemize}
\end{itemize}
\subsection{Estimation}
\begin{itemize}
    \itemsep-0.5em
    \item The number of divisors of $n$ is at most around $100$ for $n<5e4$, $500$ for $n<1e7$, $2000$ for $n<1e10$, $200000$ for $n<1e19$.
    \item The number of ways of writing $n$ as a sum of positive integers, disregarding the order of the summands. $1, 1, 2, 3, 5, 7, 11, 15, 22, 30$ for $n=0\sim 9$, $627$ for $n=20$, $\sim 2e5$ for $n=50$, $\sim 2e8$ for $n=100$.
    \item Total number of partitions of $n$ distinct elements: $B(n)=1, 1, 2, 5, 15, 52, 203, 877, 4140, 21147, 115975, 678570, 4213597,$\\
    $27644437, 190899322, \ldots$.
\end{itemize}
\subsection{General Purpose Numbers}
\begin{itemize}
\item Bernoulli numbers

$B_0=1,B_1^{\pm}=\pm\frac{1}{2},B_2=\frac{1}{6},B_3=0$

$\displaystyle\sum_{j=0}^m\binom{m+1}{j}B_j=0$, EGF is $B(x) = \frac{x}{e^x - 1}=\displaystyle\sum_{n=0}^\infty B_n\frac{x^n}{n!}$.

$S_m(n)=\displaystyle\sum_{k=1}^nk^m=\frac{1}{m+1}\sum_{k=0}^m\binom{m+1}{k}B^{+}_kn^{m+1-k}$

\item Stirling numbers of the second kind
Partitions of $n$ distinct elements into exactly $k$ groups. 

$S(n, k) = S(n - 1, k - 1) + kS(n - 1, k), S(n, 1) = S(n, n) = 1$

$S(n, k) = \frac{1}{k!}\sum_{i=0}^{k}(-1)^{k-i}{k \choose i}i^n$

$x^n     = \sum_{i=0}^{n} S(n, i) (x)_i$

\item Pentagonal number theorem

$\displaystyle\prod_{n=1}^{\infty}(1-x^n)=1+\sum_{k=1}^{\infty}(-1)^k\left(x^{k(3k+1)/2} + x^{k(3k-1)/2}\right)$

% \item Catalan numbers

% $C^{(k)}_n = \displaystyle \frac{1}{(k - 1)n + 1}\binom{kn}{n}$

% $C^{(k)}(x) = 1 + x [C^{(k)}(x)]^k$

\item Eulerian numbers

Number of permutations $\pi \in S_n$ in which exactly $k$ elements are greater than the previous element. $k$ $j$:s s.t. $\pi(j)>\pi(j+1)$, $k+1$ $j$:s s.t. $\pi(j)\geq j$, $k$ $j$:s s.t. $\pi(j)>j$.

$E(n,k) = (n-k)E(n-1,k-1) + (k+1)E(n-1,k)$

$E(n,0) = E(n,n-1) = 1$

$E(n,k) = \sum_{j=0}^k(-1)^j\binom{n+1}{j}(k+1-j)^n$

\end{itemize}
\subsection{Tips for Generating Funtion}
\begin{itemize}
\item Ordinary Generating Function
$A(x) = \sum_{i\ge 0} a_ix^i$
\begin{itemize}
    \itemsep-0.5em
    \item $A(rx)             \Rightarrow r^na_n$
    \item $A(x) + B(x)       \Rightarrow a_n + b_n$
    \item $A(x)B(x)          \Rightarrow \sum_{i=0}^{n} a_ib_{n-i}$
    \item $A(x)^k            \Rightarrow \sum_{i_1+i_2+\cdots+i_k=n} a_{i_1}a_{i_2}\ldots a_{i_k}$
    \item $xA(x)'            \Rightarrow na_n$
    \item $\frac{A(x)}{1-x}  \Rightarrow \sum_{i=0}^{n} a_i$
\end{itemize}
\item Exponential Generating Function
$A(x) = \sum_{i\ge 0} \frac{a_i}{i!}x_i$
\begin{itemize}
    \itemsep-0.5em
    \item $A(x) + B(x)       \Rightarrow a_n + b_n$
    \item $A^{(k)}(x)        \Rightarrow a_{n+k}$
    \item $A(x)B(x)          \Rightarrow \sum_{i=0}^{n} \binom{n}{i}a_ib_{n-i}$
    \item $A(x)^k            \Rightarrow \sum_{i_1+i_2+\cdots+i_k=n} \binom{n}{i_1, i_2, \ldots, i_k}a_{i_1}a_{i_2}\ldots a_{i_k}$
    \item $xA(x)             \Rightarrow na_n$
\end{itemize}
\item Special Generating Function
\begin{itemize}
    \itemsep-0.5em
    \item $(1+x)^n           = \sum_{i\ge 0} \binom{n}{i}x^i$
    \item $\frac{1}{(1-x)^n} = \sum_{i\ge 0} \binom{i}{n-1}x^i$
\end{itemize}
\end{itemize}

\section{Polynomial}
\subsection{Number Theoretic Transform}
\lstinputlisting{Polynomial/NTT.cpp}
\subsection{Fast Fourier Transform}
\lstinputlisting{Polynomial/FFT.cpp}
\subsection{Primes}
% \normalsize
\begin{tabular}{ l l l l }
    Prime & Root & Prime & Root \\
    7681  &   17 & 167772161 & 3 \\
    12289 & 11  &104857601 & 3 \\
    40961 & 3 & 985661441 & 3 \\
    65537 & 3 & 998244353 & 3 \\   
    786433 & 10 & 1107296257 & 10 \\
    5767169 & 3 & 2013265921 & 31 \\
    7340033 & 3 & 2810183681 & 11 \\
    23068673 & 3 & 2885681153 & 3 \\
    469762049 & 3 & 605028353 & 3 \\
    2061584302081 & 2748779069441 & 3 \\
    1945555039024054273 & 5 & 9223372036737335297 & 3
\end{tabular}
\subsection{Polynomial Operations}
\lstinputlisting{Polynomial/Operation.cpp}
\subsection{Fast Linear Recursion}
\lstinputlisting{Polynomial/FastLinearRecursion.cpp}
\subsection{Fast Walsh Transform}
\lstinputlisting{Polynomial/Fwt.cpp}

\section{Geometry}
\subsection{Basic}
\lstinputlisting{Geometry/Basic.cpp}
\subsection{Heart}
\lstinputlisting{Geometry/Heart.cpp}
\subsection{External Bisector}
\lstinputlisting{Geometry/external_bisector.cpp}
\subsection{Intersection of Segments}
\lstinputlisting{Geometry/Intersection_of_Segments.cpp}
\subsection{Intersection of Circle and Line}
\lstinputlisting{Geometry/Intersection_of_Circle_and_Line.cpp}
\subsection{Intersection of Circles}
\lstinputlisting{Geometry/Intersection_of_Circles.cpp}
\subsection{Intersection of Polygon and Circle}
\lstinputlisting{Geometry/Intersection_of_Polygon_and_Circle.cpp}
\subsection{Tangent Lines of Circle and Point}
\lstinputlisting{Geometry/Tangent_Lines_of_Circle_and_Point.cpp}
\subsection{Tangent Lines of Circles}
\lstinputlisting{Geometry/Tangent_Lines_of_Circles.cpp}
\subsection{Point In Convex}
\lstinputlisting{Geometry/PointInConvex.cpp} % ABC 296G
\subsection{Point Segment Distance}
\lstinputlisting{Geometry/PointSegDist.cpp}
\subsection{Convex Hull}
\lstinputlisting{Geometry/ConvexHull.cpp}
\subsection{Convex Hull Distance}
\lstinputlisting{Geometry/ConvexHullDist.cpp}
\subsection{Minimum Enclosing Circle}
\lstinputlisting{Geometry/Minimum_Enclosing_Circle.cpp}
\subsection{Union of Circles}
\lstinputlisting{Geometry/Union_of_Circles.cpp} % TIOJ 1503
\subsection{Union of Polygons}
\lstinputlisting{Geometry/Union_of_Polygons.cpp} % CF 107E, CF 101673A
% \subsection{Polar Angle Sort*}
% \lstinputlisting{Geometry/PolarAngleSort.cpp}
% \subsection{Rotating Caliper*}
% \lstinputlisting{Geometry/RotatingCaliper.cpp}
\subsection{Rotating SweepLine}
\lstinputlisting{Geometry/RotatingSweepLine.cpp}
\subsection{Half Plane Intersection}
\lstinputlisting{Geometry/HalfPlaneIntersection.cpp} % Kattis Big Brother
\subsection{Minkowski Sum}
\lstinputlisting{Geometry/MinkowskiSum_Linear.cpp} % CF 1841F
\subsection{Delaunay Triangulation}
\lstinputlisting{Geometry/DelaunayTriangulation.cpp}
\subsection{Triangulation Vonoroi}
\lstinputlisting{Geometry/Triangulation_Vonoroi.cpp}
\subsection{3D Point}
\lstinputlisting{Geometry/3DPoint.cpp}
\subsection{3D Convex Hull}
\lstinputlisting{Geometry/3DConvexHull.cpp}

\section{Else}
\subsection{Pbds}
\lstinputlisting{Else/Magic.cpp}
\subsection{Bit Hack}
\lstinputlisting{Else/BitHack.cpp}
\subsection{Dynamic Programming Condition}
\subsubsection{Totally Monotone (Concave/Convex)}
$\forall i < i^\prime, j < j^\prime$, $B[i][j] \leq B[i^\prime][j] \implies B[i][j^\prime] \leq B[i^\prime][j^\prime]$ \\
$\forall i < i^\prime, j < j^\prime$, $B[i][j] \geq B[i^\prime][j] \implies B[i][j^\prime] \geq B[i^\prime][j^\prime]$
\vspace{-1.5em}
\subsubsection{Monge Condition (Concave/Convex)}
$\forall i < i^\prime, j < j^\prime$, $B[i][j] + B[i^\prime][j^\prime] \geq B[i][j^\prime] + B[i^\prime][j]$ \\
$\forall i < i^\prime, j < j^\prime$, $B[i][j] + B[i^\prime][j^\prime] \leq B[i][j^\prime] + B[i^\prime][j]$
\vspace{-1.5em}
\subsubsection{Optimal Split Point}
If
\[ B[i][j] + B[i + 1][j + 1] \geq B[i][j + 1] + B[i + 1][j] \]
then
\[ H_{i, j - 1} \leq H_{i, j} \leq H_{i + 1, j} \]
\subsection{Smawk Algorithm}
\lstinputlisting{Else/Smawk.cpp}
\subsection{Slope Trick}
\lstinputlisting{Else/SlopeTrick.cpp}
% \subsection{Manhattan MST}
% \lstinputlisting{Else/ManhattanMST.cpp}
% \subsection{Dynamic MST}
% \lstinputlisting{Else/DynamicMST.cpp}
\subsection{ALL LCS}
\lstinputlisting{Else/All_LCS.cpp}
\subsection{Hilbert Curve}
\lstinputlisting{Else/Hilbert.cpp}
\subsection{Random}
\lstinputlisting{Else/Random.cpp}
% \subsection{Two Dimension Add Sum*}
% \lstinputlisting{Else/TwoDimensionAddAndSum.cpp}
\subsection{Matroid Intersection}
Start from $S = \emptyset$. In each iteration, let 
\vspace{-0.5em}
\begin{itemize}
    \itemsep-0.5em
  \item $Y_1 = \{x \not\in S \mid S \cup \{x\} \in I_1 \}$
  \item $Y_2 = \{x \not\in S \mid S \cup \{x\} \in I_2 \}$
\end{itemize}
If there exists $x \in Y_1 \cap Y_2$, insert $x$ into $S$. Otherwise for each $x \in S, y \not\in S$, create edges
\vspace{-0.5em}
\begin{itemize}
    \itemsep-0.5em
  \item $x \to y$ if $S - \{x\} \cup \{y\} \in I_1$.
  \item $y \to x$ if $S - \{x\} \cup \{y\} \in I_2$.
\end{itemize}
Find a \textit{shortest} path (with BFS) starting from a vertex in $Y_1$ and ending at a vertex in $Y_2$ which doesn't pass through any other vertices in $Y_2$, and alternate the path. The size of $S$ will be incremented by 1 in each iteration. For the weighted case, assign weight $w(x)$ to vertex $x$ if $x \in S$ and $-w(x)$ if $x \not\in S$. Find the path with the minimum number of edges among all minimum length paths and alternate it.

\subsection{Python Misc}
\lstinputlisting{Else/python.py}

\end{document}